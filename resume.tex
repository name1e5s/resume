\documentclass{resume}

\newcommand{\en}[1]{#1}
\newcommand{\zh}[1]{}

\zh{\usepackage{xeCJK}}
\zh{\setCJKmainfont{Source Han Serif SC}}
\zh{\setCJKsansfont{Source Han Sans SC}}
\zh{\setCJKmonofont{Source Han Sans SC}}

\begin{document}

\name{\en{Haixin Yu}\zh{于海鑫}}
\basicInfo{
      \email{me@hai-hs.in} \textperiodcentered\
      \phone{151-0113-5718} \textperiodcentered\
      \github[Hai-Hsin]{https://github.com/name1e5s} \textperiodcentered\
      \homepage[blog.hai-hs.in]{https://blog.hai-hs.in/}
}

\section{\en{Education}\zh{教育经历}}
\en{\datedsubsection{\textbf{Beijing University of Posts and Telecommunications}, Master of Computer Technology}{2021 -- 2024}}
\zh{\datedsubsection{\textbf{北京邮电大学}, 硕士}{2021 -- 至今}}
\begin{itemize}
      \item \en{Major: Computer Technology, \textit{School of Computer Science}}
            \zh{计算机学院(国家示范性软件学院),计算机技术专业}
\end{itemize}

\en{\datedsubsection{\textbf{Beijing University of Posts and Telecommunications}, Bachelor's Degree}{2017 -- 2021}}
\zh{\datedsubsection{\textbf{北京邮电大学}, 本科}{2017 -- 2021}}
\begin{itemize}
      \item \en{Major: Computer Science and Technology , \textit{School of Computer Science}, GPA: 3.50}
            \zh{计算机学院(国家示范性软件学院),计算机科学与技术专业,GPA: 3.50}
\end{itemize}

\section{\en{Work Experience}\zh{工作经历}}
\en{\datedsubsection{\textbf{\href{https://www.bytedance.com/}{ByteDance}}, Beijing, China}{09/2021 -- Now}}
\zh{\datedsubsection{\textbf{\href{https://www.bytedance.com/}{字节跳动}}}{09/2021 -- 至今}}
\en{\role{Lark Cross Platform Infrastructure}{R\&D Intern, Rust}}
\zh{\role{飞书跨平台基础架构}{Rust 研发实习}}
\begin{itemize}
      \item \en{
                Collaborated in the development of the cross-platform Rust SDK for Lark Messenger's client application.
            }
            \zh{协作开发飞书客户端跨平台 Rust SDK。}
      \item \en{
                % Maintain \textit{Secret Chat}, end-to-end encryption chat in Lark Messenger.
                % Implemented and integrated a new protocol combined with an extended 
                % Diffie-Hellman and a double-ratchet algorithm, provides post-compromise 
                % security, forward secery and resilience for Secret Chat.
                Provided security components for Lark Messenger to ensure the security of messages.
                }
            \zh{
                %维护飞书端到端加密聊天功能 “密聊”。实现并集成了一套新加密协议,通过结合扩展的 
                %Diffie-Hellman 算法以及双棘轮算法,保证了聊天的后向安全性、前向安全性以及抵抗重放攻击的能力。
                为飞书消息业务提供安全组件,确保消息的安全性。
            }
      \item \en{
                Maintained storage component of Lark Messenger, including:
            }
            \zh{
                维护飞书 SDK 的存储组件,包括:
            }
      \begin{itemize}
                  \item \en{
                              Developed \textit{squam}, a SQL toolkit based on sqlite3 focused on binary size and performance optimization.
                              Comes with a better binary footprint compared to \texttt{diesel}.
                        }
                        \zh{
                              开发了基于 sqlite3 的 SQL 工具库 \textit{squam},专注于优化二进制大小和性能。在测试中,相较于 \texttt{diesel},\textit{squam} 的二进制体积更小,编译时间更短。
                        }
                  \item \en{
                              Built an asynchronous sqlite connection pool that has higher throughput than \texttt{r2d2}.
                        }
                        \zh{
                              开发了一个异步的 sqlite 连接池,相较于 \texttt{r2d2} 吞吐量更高。
                        }
                  \item \en{
                              Introduced type-checking for SQL queries to detect most bugs at compile-time.
                        }
                        \zh{
                              引入了 SQL 查询语句的类型检查机制,在编译时能发现大部分 bug。
                        }
                  \item \en{
                              Tuned SQLite performance, maintained database-related CI, 
                              and assisted business teams in troubleshooting database issues.
                        }
                        \zh{
                              进行 sqlite 调优,维护数据库相关 CI,协助业务方排查数据库相关的问题。
                        }
            \end{itemize}
      \item \en{
                Responsible for the stability of the Lark Messenger. Troubleshoot \textit{panics} and \textit{crashes}.
                Discovered bugs and design flaws in foundational libraries such as \texttt{libunwind}, \texttt{rust std} and \texttt{darwin}.
            }
            \zh{
                负责飞书客户端的稳定性,包括定位并修复 \textit{panic} 和 \textit{crash}。
                在定位过程中发现了 \texttt{libunwind}、\texttt{rust std} 和 \texttt{darwin} 等基础库的一些 bug 和设计缺陷。
            }
      \item \en{
                Refactored flow log uploading module in Lark Messenger, improving code quality and readibility.
            }
            \zh{
                重构了飞书的日志上传模块,提升了代码质量和可读性。
            }
\end{itemize}

\en{\datedsubsection{\textbf{\href{https://www.bytedance.com/}{ByteDance}}, Beijing, China}{04/2020 -- 07/2020}}
\zh{\datedsubsection{\textbf{\href{https://www.bytedance.com/}{字节跳动}}}{04/2020 -- 07/2020}}
\en{\role{Lark iOS}{R\&D Intern, iOS}}
\zh{\role{飞书}{iOS 研发实习}}
\begin{itemize}
      \item \en{Participated in the development of Lark iOS app.}
            \zh{参与飞书 iOS 端业务研发,负责开发新功能和维护现有功能。}
      \item \en{
                  Implemented multiple feature enhancements, 
                  including adding support for selecting all group members when buzzing, etc.
            }
            \zh{实现了多项功能改进,包括提升文件传输大小限制、群加急选人支持全选等功能。}
      \item \en{Fixied various layout issues in i18n scenarios, thereby improving user experience.}
            \zh{参与品质优化相关项目,修复了多个国际化场景下 iOS 端的布局问题,提升了用户体验。}
\end{itemize}

\en{\datedsubsection{\textbf{Research Institute for Information Technology}, Tsinghua University}{10/2019 -- 04/2020}}
\zh{\datedsubsection{\textbf{清华大学信息技术研究院}}{10/2019 -- 04/2020}}
\en{\role{NSLab}{Research Intern}}
\zh{\role{网络安全实验室}{科研实习}}
\begin{itemize}
      \item \en{Explored the hardening of distributed graph database \textit{nebula} with Intel SGX.}
            \zh{探索使用 Intel SGX 技术对分布式图数据库 \textit{nebula} 进行加固。}
      \item \en{Leveraged \texttt{avx2} to accelerate the aggregation step of federated learning with SGX.}
            \zh{利用 \texttt{avx2} 指令集加速了 SGX 技术下联邦学习的模型聚合。}
\end{itemize}

\section{\en{Portfolios}\zh{个人项目}}
\begin{itemize}[parsep=0.25ex]
      \item \textbf{Cecike}:
            \en{
                An \textbf{out-of-order superscalar} RV64IMAC microprocessor core that runs on FPGA.
                Designed from scratch, the core can decode two instructions per cycle and execute four instructions per cycle.
                A GShare branch predictor is introduced to improve performance, and the final IPC is 1.6 when running test programs.
            }
            \zh{
                运行于 FPGA 上的\textbf{乱序超标量} RV64IMAC 微处理器核。
                从零开始设计的乱序双发射处理器架构,每周期最多可以同时解码两条指令,执行四条指令。
                在前端引入 GShare 分支预测器以提升性能,最终运行测试程序时的 IPC 为 1.6。}
      \item \textbf{\href{https://github.com/name1e5s/MuddyDNS}{Muddy DNS}}:
            \en{
                A DNS relay server with DNS forwarding and malicious website blocking, implemented in Go.
                It parses DNS packet headers according to RFC 1035, and operates on the packet based on thr rules.
                Supports high concurrency.
            }
            \zh{支持 DNS 转发及不良网址拦截的中继服务器,使用 Go 实现。按照 RFC 1035 所述规范解析 DNS 包头,根据处理后的结果对 DNS 包进行操作,支持高并发。}
      \item \textbf{\href{https://github.com/SpinalHDL/SpinalHDL}{SpinalHDL}}:
            \en{
                A new hardware description language for FPGA and ASIC design.
                Implemented support for using two industrial simulation suites, 
                VCS and Xilinx Vivado, as testing backends, which has received widespread praise.
            }
            \zh{SpinalHDL 是新一代的硬件语言。为其实现了将 VCS 以及 Xilinx Vivado 两大工业仿真套件作为测试后端,得到广泛好评。}
\end{itemize}

\section{\en{Honors and Awards}\zh{获奖情况}}
\en{
      \datedline{ByteDance 2022 Q3 \textbf{Spot Bonus}}{October 2022}
      \datedline{\textbf{First Prize} of the Third National Student Computer System Capability Challenge}{August 2019}
}
\zh{
      \datedline{字节跳动 2022 年第三季度 \textbf{Spot Bonus}}{2022 年 10 月}
      \datedline{“龙芯杯”第三届全国大学生计算机系统能力培养大赛\textbf{一等奖}}{2019 年 08 月}
}
\section{\en{Skills}\zh{技能}}
\begin{itemize}[parsep=0.25ex]
      \item \en{\textbf{Programming Languages}:
                  not limited to any specific language,
                  but experienced in Rust/C/C++/Chisel,
                  comfortable with Python/Swift/Go/Assembly.}
            \zh{\textbf{编程语言}:
                  不局限于特定编程语言,熟悉 Rust/C/C++/Chisel 等,
                  了解 Python/Swift/Go/Assembly 等。}

      \item \en{\textbf{Languages}:
                  Passed CET-6, can handle daily communication and document reading in English.
            }
            \zh{\textbf{语言}:
                  通过英语六级(525分),可以使用英语进行日常交流以及技术文档的阅读。
            }
      
      \item \en{\textbf{Developing Tools}:
                  experienced in Linux programming,
                  have experience with collaboration tools like Git, etc.}
            \zh{\textbf{开发工具}:
                  熟悉 Linux,有使用 Git 等协作工具的经验。}
            
      \item \en{\textbf{Open-source Contributions}:
                  contributed to \textbf{rust-lang/rust}, \textbf{SpinalHDL} etc.}
            \zh{\textbf{开源贡献}:
                  为 \textbf{rust-lang/rust}, \textbf{SpinalHDL} 等项目贡献过代码。}
\end{itemize}

\end{document}
