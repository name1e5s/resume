\documentclass{resume}

\usepackage{xeCJK}
\setCJKmainfont[ItalicFont=STKaiti]{Source Han Serif SC}
\setCJKsansfont{Source Han Sans SC}
\setCJKmonofont{Source Han Sans SC}

\begin{document}

\name{于海鑫}
\basicInfo{
      \email{me@hai-hs.in} \textperiodcentered\
      \phone{+86 151-0113-5718} \textperiodcentered\
      \github[Hai-Hsin]{https://github.com/name1e5s} \textperiodcentered\
      \homepage[blog.hai-hs.in]{https://blog.hai-hs.in/}
}

\section{教育经历}
\datedsubsection{\textbf{北京邮电大学}, 硕士}{2021 -- 2024}
\begin{itemize}
      \item 计算机学院(国家示范性软件学院),计算机技术专业,导师\ \textbf{周锋}\ 教授
\end{itemize}

\datedsubsection{\textbf{北京邮电大学}, 本科}{2017 -- 2021}
\begin{itemize}
      \item 计算机学院(国家示范性软件学院),计算机科学与技术专业,GPA 3.50
\end{itemize}

\section{工作经历}

\datedsubsection{\textbf{\href{https://www.aliyun.com/}{阿里云}}}{07/2024 -- \ \ 至今\ \ }
\role{基础平台研发工程师}{弹性计算-神龙虚拟化}
\begin{itemize}
      \item 参与阿里云 ECS 实例的研发与架构演进,修复线上问题,提升系统稳定性和可靠性。
      \item 牵头完成 ARM 与 x86 平台 QEMU 代码库的合并与上游版本升级,清理历史遗留技术债务,统一代码基线,显著降低后续维护成本与潜在风险。
      \item 基于高版本 Linux 内核适配虚拟化内核组件并验证,为下一代实例研发奠定基础。
      \item 通过对编译参数与代码逻辑的优化,将满载场景下 QEMU 内存占用(PSS 指标)大幅降低,提升单物理机 ECS 实例部署密度,增强资源受限场景下的服务能力。
\end{itemize}

\datedsubsection{\textbf{\href{https://www.aliyun.com/}{阿里云}}}{05/2023 -- 09/2023}
\role{虚拟化研发实习}{弹性计算-神龙虚拟化}
\begin{itemize}
      \item 参与 ARM64 架构神龙 Hypervisor 的研发与维护。
      \item 为神龙架构适配 ARMv8 RAS 虚拟化支持,优化虚拟机可靠性。
      \item 排查并修复线上问题,分析并优化虚拟化性能开销。
\end{itemize}

\datedsubsection{\textbf{\href{https://www.bytedance.com/}{字节跳动}}}{09/2021 -- 04/2023}
\role{Rust 研发实习}{Lark Cross Platform Infrastructure}
\begin{itemize}
      \item 参与飞书客户端跨平台 Rust SDK 的开发与维护。
      \item 为飞书消息业务实现端到端加密协议,并通过 fuzz 保证协议实现的正确性,确保消息的传输安全。
      \item 开发并优化基于 SQLite 的存储组件,设计实现轻量级 SQL 工具库 \textit{squam} 以替换 \texttt{diesel},引入编译期 SQL 语句检查机制,在保证 SQL 语句正确的前提下简化开发复杂度,提升编译效率与运行时性能。
      \item 维护客户端稳定性,修复 \texttt{panic} 与 \texttt{crash},发现并上报基础库(\texttt{libunwind}、\texttt{rust std} 等)中的缺陷。
\end{itemize}

\datedsubsection{\textbf{\href{https://www.bytedance.com/}{字节跳动}}}{04/2020 -- 07/2020}
\role{iOS 研发实习}{飞书}
\begin{itemize}
      \item 参与飞书 iOS 客户端功能开发与国际化布局优化。
\end{itemize}

\section{个人项目}
\begin{itemize}[parsep=0.25ex]
      \item \githubHref{https://github.com/name1e5s/Sirius}{Sirius}:
            基于 FPGA 的顺序双发射 MIPS 处理器核,支持完整 SoC 与外设,可运行 uCore 教学操作系统。
      \item \githubHref{https://github.com/SpinalHDL/SpinalHDL}{SpinalHDL}:
            为硬件描述语言 SpinalHDL 实现 VCS 和 Vivado 仿真后端支持,提升工业流程兼容性。
      \item \githubHref{https://github.com/name1e5s/WhereToSleep}{WhereToSleep}:
            北京邮电大学空教室查询系统,在 2020 年 - 2022 年广泛服务于海淀校区学生。
\end{itemize}

\section{获奖情况}

\datedline{阿里云弹性计算 FY25 年度 \textbf{最佳新人奖}}{2025 年 05 月}
\datedline{字节跳动 2022 年第三季度 \textbf{Spot Bonus}}{2022 年 10 月}
\datedline{“龙芯杯”第三届全国大学生计算机系统能力培养大赛\textbf{一等奖}}{2019 年 08 月}
\datedline{校一、二、三等奖学金若干}{2018 年 -- 2023 年}

\section{技能}
\begin{itemize}[parsep=0.25ex]
      \item \textbf{语言}:
                  通过英语六级(525分),可以使用英语进行日常交流以及技术文档的阅读。
      \item \textbf{编程语言}:
                  不局限于特定编程语言,可在短期内了解并编写大部分编程语言。
                  熟悉 C/Python/Rust 等,了解 Chisel/Scala/Go/汇编语言 等。
      \item \textbf{技术栈}:
                  熟悉虚拟化技术(KVM、QEMU、LibVirt);了解客户端 native 开发与优化(C/C++/Rust,全平台);了解处理器微结构设计与 FPGA 开发流程。
      \item \textbf{软技能}:
                  极为突出的系统调试和问题分析定位能力;自我驱动,能够主动推动技术改进;乐于自动化所有需要多次执行的工作。
      \item \textbf{开源贡献}:
                  为 \textbf{SpinalHDL}, \textbf{linux/perf}, \textbf{rust-lang/rust} 等项目贡献过代码。
\end{itemize}

\end{document}
