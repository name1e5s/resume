\documentclass{resume}

\newcommand{\en}[1]{#1}
\newcommand{\zh}[1]{}

\zh{\usepackage{xeCJK}}
\zh{\setCJKmainfont[ItalicFont=STKaiti]{Source Han Serif SC}}
\zh{\setCJKsansfont{Source Han Sans SC}}
\zh{\setCJKmonofont{Source Han Sans SC}}

\begin{document}

\name{\en{Haixin Yu}\zh{于海鑫}}
\basicInfo{
      \email{me@hai-hs.in} \textperiodcentered\
      \phone{+86 151-0113-5718} \textperiodcentered\
      \github[Hai-Hsin]{https://github.com/name1e5s} \textperiodcentered\
      \homepage[blog.hai-hs.in]{https://blog.hai-hs.in/}
}

\section{\en{Education}\zh{教育经历}}
\en{\datedsubsection{\textbf{Beijing University of Posts and Telecommunications}, Master of Computer Technology}{2021 -- 2024}}
\zh{\datedsubsection{\textbf{北京邮电大学}, 硕士}{2021 -- 至今}}
\begin{itemize}
      \item \en{Major: Computer Technology, \textit{School of Computer Science}}
            \zh{计算机学院(国家示范性软件学院),计算机技术专业}
\end{itemize}

\en{\datedsubsection{\textbf{Beijing University of Posts and Telecommunications}, Bachelor's Degree}{2017 -- 2021}}
\zh{\datedsubsection{\textbf{北京邮电大学}, 本科}{2017 -- 2021}}
\begin{itemize}
      \item \en{Major: Computer Science and Technology , \textit{School of Computer Science}, GPA: 3.50}
            \zh{计算机学院(国家示范性软件学院),计算机科学与技术专业,GPA: 3.50}
\end{itemize}

\section{\en{Work Experience}\zh{工作经历}}
\en{\datedsubsection{\textbf{\href{https://www.aliyun.com/}{Alibaba Cloud}}}{05/2023 -- 09/2023}}
\zh{\datedsubsection{\textbf{\href{https://www.aliyun.com/}{阿里云}}}{05/2023 -- 09/2023}}
\en{\role{Virtualization Engineer Intern}{X-Dragon Team, ECS}}
\zh{\role{虚拟化研发实习}{弹性计算-神龙虚拟化}}
\begin{itemize}
      \item \en {
            Participated in DragonFly Hypervisor development for Alibaba Cloud ECS on ARM64 architecture.
      }
      \zh {
            参与 ARM64 架构、倚天平台上神龙 Hypervisor 的研发。
      }
      \item \en {
            Implemented the ARMv8 RAS virtualization under DragonFly, improving VM availability under hardware errors.
      }
      \zh {
            在神龙架构下,实现了 ARMv8 RAS 功能扩展的虚拟化,提高了虚拟机在硬件错误下的可用性。
      }
      \item \en {
            Investigated and fixed online issues. Analyzed and optimized virtualization performance overhead.
      }
      \zh {
            线上问题的排查与修复。虚拟机性能开销的分析与优化。
      }
\end{itemize}
\en{\datedsubsection{\textbf{\href{https://www.bytedance.com/}{ByteDance}}}{09/2021 -- 04/2023}}
\zh{\datedsubsection{\textbf{\href{https://www.bytedance.com/}{字节跳动}}}{09/2021 -- 04/2023}}
\en{\role{Rust R\&D Intern}{Lark Cross Platform Infrastructure}}
\zh{\role{Rust 研发实习}{Lark Cross Platform Infrastructure}}
\begin{itemize}
      \item \en{
                Collaborated in the development of the cross-platform Rust SDK for Lark Messenger's client application.
            }
            \zh{协作开发飞书客户端跨平台 Rust SDK。}
      \item \en{
                Provided security components for Lark Messenger to ensure the security of messages.
                }
            \zh{
                为飞书消息业务提供安全组件,确保消息的传输安全。
            }
      \item \en{
                Maintained storage component of Lark Messenger, including:
            }
            \zh{
                维护飞书 SDK 的存储组件,包括:
            }
      \begin{itemize}
                  \item \en{
                              Developed \textit{squam}, a SQL toolkit based on sqlite3 focused on binary size and performance optimization.
                              Comes with a better binary footprint compared to \texttt{diesel}.
                        }
                        \zh{
                              开发了基于 sqlite3 的 SQL 工具库 \textit{squam},专注于优化二进制大小和性能。在测试中,相较于 \texttt{diesel},\textit{squam} 的二进制体积更小,编译时间更短。
                        }
                  \item \en{
                              Introduced type-checking for SQL queries to detect most bugs at compile-time.
                        }
                        \zh{
                              引入了 SQL 查询语句的类型检查机制,在编译时能发现大部分 bug。
                        }
                  \item \en{
                              Tuned SQLite performance, maintained database CI, 
                              and assisted business teams in troubleshooting database issues.
                        }
                        \zh{
                              进行 sqlite 调优,维护数据库相关 CI,协助业务方排查数据库相关的问题。
                        }
            \end{itemize}
      \item \en{
                Maintain the stability of \textit{Lark Messenger}. Troubleshoot \textit{panics} and \textit{crashes}.
                Discovered bugs and design flaws in foundational libraries such as \texttt{libunwind}, \texttt{rust std} and \texttt{darwin}.
            }
            \zh{
                维护飞书客户端的稳定性,包括定位并修复 \textit{panic} 和 \textit{crash}。
                在定位过程中发现了 \texttt{libunwind}、\texttt{rust std} 和 \texttt{darwin} 等基础库的一些 bug 和设计缺陷。
            }
\end{itemize}

\en{\datedsubsection{\textbf{\href{https://www.bytedance.com/}{ByteDance}}}{04/2020 -- 07/2020}}
\zh{\datedsubsection{\textbf{\href{https://www.bytedance.com/}{字节跳动}}}{04/2020 -- 07/2020}}
\en{\role{iOS R\&D Intern}{Lark iOS}}
\zh{\role{iOS 研发实习}{飞书}}
\begin{itemize}
      \item \en{Participated in the development of Lark iOS app.}
            \zh{参与飞书 iOS 端业务研发,负责开发新功能和维护现有功能。}
      \item \en{
                  Implemented multiple feature enhancements, 
                  including adding support for selecting all group members when buzzing, etc.
            }
            \zh{实现了多项功能改进,包括提升文件传输大小限制、群加急选人支持全选等功能。}
      \item \en{Fixed various layout issues in i18n scenarios, thereby improving user experience.}
            \zh{参与品质优化相关项目,修复了多个国际化场景下 iOS 端的布局问题,提升了用户体验。}
\end{itemize}

\en{\datedsubsection{\textbf{Research Institute for Information Technology}, Tsinghua University}{10/2019 -- 04/2020}}
\zh{\datedsubsection{\textbf{清华大学信息技术研究院}}{10/2019 -- 04/2020}}
\en{\role{Research Intern}{NSLab}}
\zh{\role{科研实习}{网络安全实验室}}
\begin{itemize}
      \item \en{Explored the hardening of distributed graph database \textit{nebula} with Intel SGX.}
            \zh{探索使用 Intel SGX 技术对分布式图数据库 \textit{nebula} 进行加固。}
      \item \en{Leveraged \texttt{avx2} to accelerate the aggregation step of federated learning with SGX.}
            \zh{利用 \texttt{avx2} 指令集加速了 SGX 技术下联邦学习的模型聚合。}
\end{itemize}

\section{\en{Portfolios}\zh{个人项目}}
\begin{itemize}[parsep=0.25ex]
      \item \githubHref{https://github.com/name1e5s/Sirius}{Sirius}:
            \en{
                An in-order \textbf{dual-issue} microprocessor core that runs on FPGA.
                Additionally, a complete SoC has been built based on this processor, supporting peripherals such as Ethernet and serial ports.
                The uCore teaching operating system can run on the SoC, and it's theoretically capable of running a complete Linux operating system.
            }
            \zh{
                运行于 FPGA 上的\textbf{顺序双发射} MIPS 微处理器。
                此外,基于该处理器搭建了完整的 SoC,支持网口、串口等外设。SoC 上可以运行 uCore 教学用操作系统,理论上能够运行完整的 Linux 操作系统。
            }
      \item \githubHref{https://github.com/SpinalHDL/SpinalHDL}{SpinalHDL}:
            \en{
                A new hardware description language for FPGA and ASIC design.
                Implemented support for using two industrial simulation suites, 
                VCS and Xilinx Vivado, as testing backends, which has received widespread praise.
            }
            \zh{SpinalHDL 是新一代的硬件语言。为其实现了将 VCS 以及 Xilinx Vivado 两大工业仿真套件作为测试后端,得到广泛好评。}
\end{itemize}

\section{\en{Honors and Awards}\zh{获奖情况}}
\en{
      \datedline{ByteDance 2022 Q3 \textbf{Spot Bonus}}{October 2022}
      \datedline{\textbf{First Prize} of the Third National Student Computer System Capability Challenge}{August 2019}
      \datedline{Several scholarships of Beijing Univ. of Posts and Telecommunications}{2018 -- 2022}
}
\zh{
      \datedline{字节跳动 2022 年第三季度 \textbf{Spot Bonus}}{2022 年 10 月}
      \datedline{“龙芯杯”第三届全国大学生计算机系统能力培养大赛\textbf{一等奖}}{2019 年 08 月}
      \datedline{校一、二、三等奖学金若干}{2018 年 -- 2022 年}
}
\section{\en{Skills}\zh{技能}}
\begin{itemize}[parsep=0.25ex]
      \item \en{\textbf{Languages}:
                  passed CET-6, can handle daily communication and document reading in English.
            }
            \zh{\textbf{语言}:
                  通过英语六级(525分),可以使用英语进行日常交流以及技术文档的阅读。
            }
      \item \en{\textbf{Programming Languages}:
                  not limited to any specific language,
                  but experienced in Rust/C/C++/Chisel,
                  comfortable with Python/Swift/Go/Assembly.
            }
            \zh{\textbf{编程语言}:
                  不局限于特定编程语言,熟悉 Rust/C/C++/Chisel 等,
                  了解 Scala/Python/Swift/Go/Assembly 等。
            }
      \item \en{\textbf{Development Skills}:
                  familiar with client side native development(C/C++/Rust) and virtualization(KVM/QEMU/LibVirt). \textbf{extremely} good at problem analysis and investigation.
            }
            \zh{\textbf{开发技能}:
                  熟悉客户端 native 开发(C/C++/Rust,全平台)以及虚拟化技术(KVM、QEMU、LibVirt)。了解处理器微结构设计与 FPGA 开发流程。分析定位问题能力尤其突出。
            }
      \item \en{\textbf{Open-source Contributions}:
                  contributed to \textbf{rust-lang/rust}, \textbf{SpinalHDL} etc.
            }
            \zh{\textbf{开源贡献}:
                  为 \textbf{rust-lang/rust}, \textbf{SpinalHDL} 等项目贡献过代码。
            }
\end{itemize}

\end{document}
